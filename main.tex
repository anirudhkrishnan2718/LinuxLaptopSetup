\documentclass[10pt,letterpaper,twocolumn]{article}
\usepackage[utf8]{inputenc}
\usepackage{amsmath}
\usepackage{amsfonts}
\usepackage{amssymb}
\usepackage{graphicx}
\usepackage{hyperref}
\usepackage[margin=0.75in]{geometry}


\title{Setting up a fresh Linux Install on a Laptop}
\author{Anirudh Krishnan}
\date{\today}
\setlength{\columnsep}{0.25in}

\begin{document}
\maketitle
\newpage

\tableofcontents
\newpage

\section{Pre Installation}

Pick a distribution. Ubuntu Used here due to extensive online help and debian being a user friendly package installer. Second preference for Linux Mint. \\

Pre-install involves deciding whether to Dual Boot with Windows or to Single boot Linux. Dual Booting with Windows requires Windows to be installed first because it does not have a Boot Manager to manage Linux alongside it. \\

Laptops may require these BIOS level settings changes for Dual Boot:

\begin{itemize}
	\item SecureBoot turned off (new laptops)
	\item Raid Mode turned off (manufacturer might have set Raid-0 mode On)
\end{itemize}

The simpler option is to solo-install Linux, which is what this document covers.

\subsection{Downloading the Linux Image}

Princeton Mathematics is the best mirror in the official list. Download either the long-term support version (released once every 4 years) or the latest stable release (releaed every 6 months). Look for the file resembling \texttt{ubuntu-YY.MM-desktop-amd64.iso} \\

The ISO file, a USB stick and an existing Windows or Linux machine are required to start creating the Bootble media.



\subsection{Bootable Media}
Among the many alternatives, (Unetbootin, LinuxLive Usb Creator etc.), Etcher by Balena is a cross platform neatly packaged app that makes bootable USBs. Download the app (AppImage format for Linux so it works across distros) and follow instructions to create a bootable USB. USB stick of size at least 4 GB required.\\

\url{https://www.balena.io/etcher/} \\

\newpage

\section{Installation}

\subsection{Installation Procedure}
All of the standard Ubuntu install options will work, except the \textbf{Installation Type} screen. The \textit{Install alongside Windows} option is often unavailable, so the \textit{Erase everything and install Ubuntu} option is the straightforward choice here. After the Install procedure finishes, a reboot is required. A successful reboot leading to the Login splash screen and then the desktop is an indicator of a successful Ubuntu Install.

\subsection{Installation problems}

\begin{itemize}
	\item Corrupted Installation media. Check the SHA256 key and verify the integrity of the ISO file after downloading. Then retry creation of bootable USB. Else, switch to a different release or a different distribution.
	\item Windows installed on top of existing Linux. This is the incorrect order of OS installation for dual booting. Refer to GRUB Recovery help to regain the normal OS selection screen upon booting.
\end{itemize}

\subsection{Picking Username and Password}
The Ubuntu Username and the institute email Username can (and should) be different. Also keep the Ubuntu password distinct from the Institute EMail password.

\newpage

\section{Software}
\subsection{Preparation - Software Management}

Use the Ubuntu Software Center to install Synaptic Package Manager. The Ubuntu Software Center comes bundled with the default Ubuntu ISO. On newer distributions, this also includes Snap support. \\

\url{https://snapcraft.io/store}\\

Almost all packages not available through the Software Center are easily found within Synaptic or installed using \texttt{apt-get} in the terminal, which is bundled with the default Ubuntu ISO. Synaptic is also capable of installing a \texttt{.deb} standalone file.

\subsection{Desktop Customization}
Install \texttt{gnome-tweaks} or \texttt{unity-tweak-tool} (depending on desktop environment) using Synaptic to enable detailed customization of:


\begin{itemize}
	\item Desktop workspaces 
	\item Input methods (keyboard shortcuts) 
	\item Dock behavior and placement
	\item Touchpad gestures and mouse emulation
	\item Top Status bar
	\item System-wide fonts (use synaptic to install new fonts) 
	\item Power management and Hibernation
\end{itemize}

\subsection{Music}

Audio editing is not a part of my requirements, and is not mentioned here.
Music apps installed using Software Center:

\begin{itemize}
	\item Rhythmbox pre-packaged
	\item Spotify through Software Center (Snap app).
\end{itemize}

\subsection{Terminal Customization}

Default GNOME Termial pre-packaged.

\begin{itemize}
	\item Terminal comes with some themes and text color schemes bundled. Further customization available through Synaptic.
	\item Drop down Quake-style terminals like Yakuake, Guake.
	\item Terminal transparency slider built into Preferences menu.
\end{itemize}

\subsection{Text Editor and IDE}
The following are available through the Software Center.

\begin{itemize}
	\item Gedit default text editor
	\item Sublime Text is a text editor. One time per-user license fee. No limit on number of installs. Indefinite evaluation period. Extensible with Python and C/C++ plugins. A Python interpreter within Sublime Text is also available through \texttt{SublimeREPL}. 
	\item Jetbrains Pycharm is a Python IDE.
	\item Jetbrains CLion is a C/C++ IDE with free (annually renewed) subscription to the Professional edition for students (institutional email required).
\end{itemize}

\subsection{MATLAB}

Mathematica is heresy.\\

Matlab on the Software Center is simply a tool to create a desktop and launcher icon for an existing Matlab installation. It is not the actual Matlab installer.
For Princeton University, detailed installation instructions for Matlab on Linux are available online. A free Mathworks account is required. Keep Mathworks login and password on hand when installing Matlab. The subscription is free for students and renews annually.\\

The MATLAB License verification procedure is also streamlined now and only requires the Mathworks login details as above. Where \texttt{sudo} is required is clearly specified in the instructions. Failure to use \texttt{sudo} leads to problems with the license verification process. When License verification asks for Linux Username, the actual username should be used and not \texttt{root}.\\

\url{https://princeton.service-now.com/snap?sys_id=KB0011359&id=kb_article}\\

The end result should be a MATLAB installation that can be called using \texttt{matlab} from the command line as well as through the app list \texttt{Super + A}. It should also be possible to pin the MATLAB shortcut to the dock.

\subsection{Typesetting}

Between texLive and MikTex distributions for the Latex base files, only texLive is available on the Synaptic package list. On Windows however, MikTex is the better option because of the attached MikTex package manager which helps download and update individual components of the latex base files.

\begin{itemize}
	\item Install \texttt{texlive-full} from Synaptic or fall back to \texttt{texlive-base} when constrained by low disk space or bad internet.
	\item Texstudio is a latex focused IDE that requires no extra configuration after install and has very good auto-complete tooltips.
	\item Evince is the built-in PDF viewer to which Texstudio can be asked to output after compiling a \texttt{.tex} file. 
	\item Okular is a full featured PDF viewer (highlighting and annotations) that requires a KDE environment (extra headache unless on Kubuntu).
\end{itemize} 


\subsection{Bibliography Management}

Zotero and Mendeley are the only convenient options in terms of browser plugins to import new items easily. Both are unavailable in the Software Center and need to be downloaded from their websites. University Library provides support for these two solutions as well as Endnote and 

\begin{itemize}
	\item Mendeley is proprietary, has a 2GB online storage limit, has a linux standalone client and a built-in PDF viewer.
	\item Zotero is open source, has a 300MB storage limit, also has a standalone client and is better at incorporating non-PDF sources.
\end{itemize}


\subsection{Version Control}

GitHub for the online storage of local git repositories is the easiest to set up.

\begin{itemize}
	\item Create a free account at \url{www.github.com}. They offer a free text-editor called ATOM for student accounts. Github requires a subscription to allow more than 3 collaborators and to enable private repositories. There is a 1GB limit on total space used and a 100MB limit on individual file sizes.
	\item Install \texttt{git} and the \texttt{git-hub} packages for command line access to github.com. There is no stand-alone linux client as of right now (There is a very nice Windows client though).
\end{itemize}

\subsection{Programming Languages}
Python, the Anaconda package which contains numpy,scipy,matplotlib, and C/C++ are the only languages covered here.

\begin{itemize}
	\item Python3 is pre-packaged along with the C/C++ language and compiler (gcc). IDEs for these languages covered above. Linux needs very minimal setup to start coding in these 2 languages. Python3 comes with a built-in barebones code editor called IDLE. 
	\item Install Anaconda from their website \\ \url{https://www.anaconda.com/distribution/}. 
	\item \texttt{conda list} to check if every component is installed
	\item \texttt{conda update conda} followed by \texttt{conda update anaconda} to update the entire suite.
	\item \texttt{anaconda-navigator} to open the centralised Anaconda Navigator to access the Spyder IDE, Jupyter notebooks and other components. There is no easy way to get an app shortcut for the Navigator onto the dock or the desktop.
\end{itemize}

\subsection{Media Player}

VLC is the one stop solution to all audio and video media with the correct codecs downloaded.

\begin{itemize}
	\item the package \texttt{ubuntu-restricted-extras} for most of the common codecs.
	\item VLC from the Software Center
\end{itemize}

\subsection{Web Browser}

\begin{itemize}
	\item Firefox comes pre-packaged. Substitues for Google Chrome very well.
	\item Google Chrome Linux installer is only available from their website. Software Center only lists the underlying open source Chromium browser.
	\item Command line web browser package \texttt{lynx}. Useful when GUI access is unavailable.
	
\end{itemize}

\subsection{Office Suite}

The Microsoft Office 365 browser-based software does work. A free MS Office 365 account is available to students with an institutional ID. No standalone Linux client for MS Office currently available. \\

The prepackaged LibreOffice suite is a good enough substitute. For collaborative work or online sharing, Google Docs or Office 365 Shared Docs can work, especially if the other collaborators are running Windows.

\subsection{Remote Access}

\begin{itemize}
	\item Remmina is the remote desktop client prepackaged with Ubuntu.
	\item install the \texttt{ssh} meta-package to connect to remote servers using the SSH protocol.
	\item Xterm and UXterm are prepackaged with Ubuntu.
	\item install the \texttt{samba, smbclient} and \texttt{cifs-utils} packages from Synaptic to access the University Fileserver (H: Drive). Instructions at Princeton Knowledge Base KB0010622.
	\item Departmental file-server mounting instructions available from the department. (require same packages as H: drive above)
	 
\end{itemize}

\subsection{Peripherals}

\begin{itemize}
	\item Wireless Mouse and Keyboard supported out of the box.
	\item Printer access does not require any new software. Only the printer URL is needed.
	\item Laptop Webcam works out of the box (\texttt{Cheese} software prepackaged with Ubuntu ISO)
	\item External Monitors are supported via HDMI cables. Maximum resolution is limited when using HDMI. This issue may depend on Laptop HDMI Port Specification and External Monitor Model.
	\item Laptop Fingerprint Reader does not work. Manufacturer has no Linux drivers in development.
	\item Wi-Fi works out of the box. No additional software needed.
	\item Ethernet may require host-registration by the department to work. No software to be downloaded.
\end{itemize}

\newpage

\section{Common Issues with Linux}

Issues are specific to a Dell XPS 15 9570 (newest model as of January 2019). Dell XPS 15 and Dell XPS 13 are the most popular Windows Laptop models purchased within the University so these issues should be fairly common. \\

Issues are observed and solutions tested on Ubuntu Linux (19.04), but are expected to pop up on other Linux distributions as well.

\subsection{Login Loop}

Entering the correct password on the login splash screen causes a blank screen and a loop back to the same login screen.\\

Solved by switching to Wayland window manager instead of X11 on the login screen. Many proposed solutions online and mane possible underlying causes of the issue. 

\subsection{Battery Life}

It is unknown how the battery life on Ubuntu compares to Windows on the same Laptop model. This depends on Linux Kernel version.

\subsection{External Monitors}

HDMI Cable is not able to support high resolution external monitors. May require a DisplayPort cable. HDMI is sufficient on Windows however. This was determined to not be a driver or software issue on Linux. The HDMI display cable is the limiting hardware.\\

Switching to a USB-C Gen 2 (aka Thunderbolt 3) to DisplayPort cable for the monitor supports 1440p@60Hz as tested. This cable can support up to 4K@60Hz according to the product description. Avoiding HDMI cables to begin with and using a USB-C Dock or a DisplayPort display cable is recommended.

\subsection{Bad Desktop Graphics Performance}

The default Gnome3 Desktop and the gdm3 window manager are very choppy when it comes to desktop graphics (resizing, dragging, minimizing windows). The animation smoothness does not match the monitor's high refresh rate. \\

Install \texttt{ubuntu-unity-desktop} and switch to Unity desktop on the login splash screen. When prompted, switch to \texttt{lightdm} from \texttt{gdm3} to recover the old Unity login splash screen. Desktop graphics and animations are far smoother in Unity. Unity was the default desktop before Ubuntu 17.10 when it switched to GNOME. Unity desktop also has a far better window tiling manager. Install \texttt{unity-tweak-tool} for finer control over the Unity desktop.\\

This issue persists when switching to Nvidia discrete graphics from the integrated Intel graphics. A smoother desktop graphics experience might be possible with a different desktop environment such as KDE or XFCE.

\end{document}